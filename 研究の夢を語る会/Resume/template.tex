\documentclass[twocolumn, a4paper, 9pt]{jarticle}

\makeatletter
\def\section{\@startsection{section}{1}{\z@}{2ex plus .2ex minus .2ex}%
  {.5ex plus .2ex minus .2ex}{\large\bfseries}}
\def\thesection{\arabic{section}.}
\def\subsection{\@startsection{subsection}{1}{\z@}{.7ex plus .2ex minus .2ex}%
  {.5ex plus .2ex minus .2ex}{\normalsize\bfseries}}
\def\thesubsection{\arabic{section}.\arabic{subsection}}
\def\thefootnote{\fnsymbol{footnote}}
\makeatother

% ipsj-kansai
\setlength{\topmargin}{-10mm} % 15mm - 1in
\setlength{\headheight}{5mm}
\setlength{\headsep}{5mm}
\setlength{\oddsidemargin}{-7mm} % 18mm - 1in
\setlength{\evensidemargin}{-7mm} % 18mm - 1in
\setlength{\textheight}{247mm} % 297 - 25(top) - 25(bottom)
\setlength{\textwidth}{174mm} % 210 - 18*2
\setlength{\columnsep}{7mm}

\usepackage{fancyhdr}
\pagestyle{empty}
\thispagestyle{fancy}
\rhead[]{\fontsize{9pt}{9pt} 研究の夢を語る会@SocSEL}
\cfoot[]{}
\renewcommand{\headrulewidth}{0pt}

\begin{document}
\twocolumn[
  \begin{center}
    {
      {\bf \Large オブジェクト操作入力に基づくビジュアルプログラミング作品検出の試み \\} % 14.4pt
      \vspace{0.5ex}
    }
    \vspace{1.5ex}
    {
      \fontsize{10.5pt}{10pt}
      \begin{tabular}{c}
        著者$^\dag$ \\
        福地 ユキ
      \end{tabular}
    }
    \vspace{1.5ex}
  \end{center}
]


%%%%%%%%%%%%%%%%%%%%
\section{はじめに}
近年,ビジュアルプログラミング言語を利用したプログラミング教育が進められている.
ビジュアルプログラミング言語は,プログラム中の命令処理をブロックで視覚的に表現し,
そのブロックを組み合わせることでプログラミングを実現している.
ビジュアルプログラミング言語の利点は,学習者が記法にとらわれずパズルのように実装できることである.
代表的なビジュアルプログラミングサービスとして,Scratchが挙げられる.
Scratchでは,制作されたプログラム作品がオンライン上に公開されている.

多くのビジュアルプログラミング言語には明確な学習順序がなく,
学習者が自由な発想で作品を制作するが故に,公開されている既存のプログラムを模倣することから学習を開始する.
しかし,Scratch作品のキーワード検索において,学習者が実装しようとする動作の言語化が困難であることや,
作品のタイトルや概要に動作に関するキーワードが含まれているとは限らないことから,
学習者の参考となる作品の検索は容易ではないといえる.

この研究では,学習者が実装しようとする動作を視覚的なオブジェクト操作を検索クエリとして,
サービスに公開される膨大なプログラム作品の中から学習者のイメージに類する作品を検出する手法を提案する.
テキストベースのプログラムでは,命令処理の完全一致によりプログラム検索手法が提案されているが,
ビジュアルプログラミングを対象に画像検索することで,実装方法の異なる類似作品の検出が期待できる.
%%%%%%%%%%%%%%%%%%%%


%%%%%%%%%%%%%%%%%%%%
\section{手法}
キー入力やマウス操作を必要としない,動きブロック(例: 指定した座標位置にオブジェクトを移動させるブロック,オブジェクトを回転させるブロックなど)
を用いた単純な動作を行う作品を対象に検索を行う.

\noindent\textbf{1.検索対象の作品を準備:}
公開された作品の動画から,Webアプリケーションの自動テストツールSeleniumを用いてスナップショットを収集し,
検索対象の各作品の最初と任意時点のフレームの画像を取得する.

\noindent\textbf{2.スプライトの移動方向を検出:}
画像認識技術のSIFT 特徴量を用いて,1. で取得した2 つの画像間で移動するオブジェクトを特定し,
移動前後の座標から移動方向を算出する.
具体的には,オブジェクトから右直線方向の
$0^\circ$,($0^\circ$〜$90^\circ$),$90^\circ$,($90^\circ$〜$180^\circ$),$180^\circ$,($180^\circ$〜$270^\circ$),$270^\circ$,($270^\circ$〜$360^\circ$)
の8 方向の移動に分類して検出する.
ただし,オブジェクトの移動距離は任意とする.

\noindent\textbf{3.入力オブジェクト操作の移動方向を検出:}
学習者が入力したオブジェクトの移動操作の最初と最後のフレームの画像に対して,
2. と同様にSIFT 特徴量を用いて,2 つの画像間で移動するオブジェクトを特定し,
移動前後の座標から移動方向を検出する.

\noindent\textbf{4.移動方向が一致する作品を推薦:}
公開作品の中から,入力オブジェクト操作の移動方向(8 方向)が一致する作品を推薦する.
%%%%%%%%%%%%%%%%%%%%

%%%%%%%%%%%%%%%%%%%%
\section{結果}
提案手法を用いて,Scratch の公開作品を実験者が制作し,
オブジェクトが画面左下から右上へ移動する作品を検出する動作実験を実施した.
提案手法を適用した結果,プログラムの内容は異なっていても移動方向が同じ作品を検出できることを確認した.
具体的には,画像の左下から右上へ直線移動,円を描きながら移動など様々なプログラムで移動する作品を検出した.

(現在進行途中の「実際の作品を対象に検索」が完了後,そちらの結果も記述したい)
%%%%%%%%%%%%%%%%%%%%

%%%%%%%%%%%%%%%%%%%%
\section{今後の課題}
現時点ではオブジェクトの移動方向のみを用いた作品検索を実現しているが,
今後は複雑な動作(例:移動や回転を組み合わせた動作など)の検出を行う.
そのためには,最初と任意時点のフレームの画像を取得するのではなく,
詳細な間隔ごとのフレームの画像を取得し,各フレーム間におけるオブジェクトの座標変化を観察することで,
動作の検出方法を検討する必要がある.

また,現時点では著者が指定した画像をオブジェクトとしているが.実際の公開作品では様々な画像がオブジェクトとして使用されているため,
動作するオブジェクトを特定する手法を検討する必要がある.

(現在進行途中の「実際の作品を対象に検索」が完了後,実際の作品を対象にした際の課題も出てくると思うため,それについても記述したい)

%%%%%%%%%%%%%%%%%%%%

\bibliographystyle{ipsjunsrt}
\bibliography{bibfile}

\end{document}